\section{Συνοπτική Επισκόπηση Προόδου}
Οι συμμετάσχοντες έχουν επικοινωνήσει με μία τοπική επιχείρηση επονομαζόμενη \\Balance Hotel και έχει διεξαχθεί η πρώτη συνάντηση με τον εκπρόσωπο του επιχείρησης. \\

\noindent
Μέσω αυτής έχουν καταγραφεί τα βασικά χαρακτηριστικά του καταλύματος:
	\begin{itemize}
		\item Αριθμός εταιριών με τις οποίες συνεργάζεται ξενοδοχείο.
		\item Τρόποι εύρεσης της επιχείρησης. 
		\item Τρόπους επικοινωνίας πελάτη με την επιχείρηση για να γίνει κράτηση.
		\item Τρόπους επικοινωνίας επιχείρησης με τους συνεργάτες.
		\item Τρόπους καταγραφής προμηθειών.
	\end{itemize}

\hspace{1cm}\\
\noindent
Επίσης, έγινε ενημέρωση για τις παροχές που προσφέρει το κατάλυμα
\begin{itemize}
	\item Πρωινό.
	\item Υπηρεσίες δωματίου (Φαγητό).
 	\item Ποδήλατα για μετακίνηση μέσα στην πόλη.
 	\item Εφαρμογές για ενοικίαση οχημάτων, προτάσεις εστιατορίων και αξιοθέατων.
 	\item Παροχές που αντιστοιχούν στην κατηγορία καταλυμάτων που βρίσκεται.
\end{itemize} 

\noindent
Δόθηκε ιδιαίτερη βάση στις παροχές που προσφέρει το κατάλυμα και ειδικότερα σε αυτές που ανεβάζουν την κατηγορία ενός καταλύματος. Πιο συγκεκριμένα, διαθέτει έξυπνες λάμπες, smart tv, έξυπνους θερμοστάτες και tablet για την χρήση από του πελάτες.\\

\noindent
Όσον αφορά τα προβλήματα που αντιμετωπίζει η επιχείρηση. Αναφέρθηκαν προβλήματά μεταξύ της επικοινωνίας των πελατών και της υποδοχής καθώς πέραν του τηλεφώνου δεν υπάρχει άλλος τρόπος άμεσης επικοινωνίας με αποτέλεσμα πολλοί από τους πελάτες να επικοινωνούν με την εκπρόσωπο που είναι και η υπάλληλος της υποδοχής μέσω εφαρμογών όπως το Whats app και το viber. Αυτό αρχικά χαρακτηρίστηκε ως αντιεπαγγελματικό και επίσης αρκετά κουραστικό για την υπάλληλο της υποδοχής καθώς τα μηνύματα έρχονται και σε ώρες εκτός του ωραρίου. Επίσης, αναφέρθηκε πως πολλές από τις ερωτήσεις που γίνονται στον υπάλληλο υποδοχής είναι ερωτήσεις που αναφέρονται ήδη στις παρεχόμενες εφαρμογές με πληροφορίες.\\


\noindent
Ακόμα, αναφέρθηκε πως η διαδικασία απογραφής των δωματίων που καθαρίζονται μετά το πέρας της διαμονής ενός πελάτη είναι αρκετά χρονοβόρα και γίνεται με έναν μη αποδοτικό τρόπο αυτή τη στιγμή. Πιο συγκεκριμένα, η καταγραφή των δωματίων που ανέλαβε κάθε υπάλληλος καθαριότητας καθώς και η απογραφή  αντικειμένων που λείπουν από αυτά, αναφέρονται στον υπάλληλο της γραμματείας ο οποίος τα καταγράφει σε ένα excel.\\

\noindent
Όσον αφορά τις επιθυμίες του πελάτη, εκφράστηκε η θέληση να γίνει ουσιαστική χρήση των tablet τα οποία διαθέτει το κατάλυμα από τους ενοίκους, καθώς στην παρούσα κατάσταση δεν συμβαίνει κάτι τέτοιο αλλά και η αναβάθμιση του τρόπου καταγραφής των δωματίων που καθάρισε κάθε υπάλληλος καθώς και των αντίστοιχων απωλειών που αναφέρουν. Αυτές ευκαιρίες παρουσιάστηκαν κατά την πρώτη σύσκεψη με τον υπεύθυνο.\\

\noindent
Γνωρίζοντας πλέον τις αδυναμίες της επιχείρησης, οι συμμετάσχοντες αποφάσισαν να υλοποιήσουν ένα σύστημα του οποίου σκοπός θα  είναι η διευκόλυνση της επικοινωνίας μεταξύ πελάτη και υποδοχής για σκοπούς απλής επικοινωνίας ή για παροχή υπηρεσιών δωματίου καθώς και για την καταγραφή των των δωματίων από τους υπαλλήλους καθαριότητας. Οι ενδεχόμενες λειτουργίες εμφανίζονται παρακάτω:\\  

\begin{itemize}
	\item (Πελάτης) Συχνές ερωτήσεις σχετικά με την πόλη, τα αξιοθέατα καθώς και για προτεινόμενες επιχειρήσεις όπως εστιατόρια, μαγαζιά με μουσική, επιχειρήσεις ενοικίασης αυτοκινήτων(σύμφωνα με προτάσεις της γραμματείας).
	\item (Πελάτης) Δυνατότητα χρήσης υπηρεσιών δωματίου όπως για παράδειγμα πρωινό στο δωμάτιο ή αγορά κάποιου άλλου φαγητού ή ποτού όπως κάποιου κρασιού.
	\item (Πελάτης) Δυνατότητα επικοινωνίας με την υποδοχή σε περίπτωση που οι παρεχόμενες πληροφορίες/δυνατότητες του συστήματος δεν επαρκούν.
	\item (Υποδοχή) Επικοινωνία με τους ένοικους. 
	\item (Υπάλληλος καθαριότητας) Απογραφή δωματίων / απωλειών. 
	\item (Υπάλληλος καθαριότητας) Απάντηση σε αιτήματα room service από πελάτες.
\end{itemize} 

\noindent
Οι συμμετάσχοντες θεωρούν πως η υλοποίηση αυτού του συστήματος θα έχει ως αποτέλεσμα την αύξηση της παραγωγικότητας της γραμματείας καθώς δεν θα υπάρχουν διασπάσεις προσοχής λόγω των συνεχών μηνυμάτων καθώς και η καταγραφή των δωματίων θα γίνεται χωρίς την ανάγκη του υπάλληλου υποδοχής. Επιπλέον, η εμπειρία διαμονής θα γίνει πολύ πιο εύκολη και πολύ πιο ευχάριστη καθώς η επικοινωνία θα είναι πολύ πιο άμεση και εύκολη.\\ 

\noindent
Όσον αφορά τα προβλήματα που έχουν παρουσιαστεί, μέχρι τώρα οι συμμετάσχοντες δεν έχουν συναντήσει κάποιο πρόβλημα πέραν της αρχικής επικοινωνίας με τον πελάτη.\\

\noindent
Τέλος, όσον αφορά τον καταμερισμό εργασίας και εφόσον η εργασία βρίσκεται σε αρχικό στάδιο, αρχικά χωρίστηκε τό πρώτο κομμάτι της εργασίας (Ενότητα 2) στους συμμετάσχοντες και στη συνέχεια, θα γίνει διαχωρισμός μίας μεγαλύτερης υποενότητας στον κάθε ένα, ώστε να γίνεται με αποδοτικό τρόπο η αναφορά.