\section{Συνοπτική Επισκόπηση Προόδου}
Οι συμμετάσχοντες έχουν επικοινωνήσει με μία τοπική επιχείρηση επονομαζόμενη \\Balance Hotel και έχει διεξαχθεί η πρώτη συνάντηση με τον εκπρόσωπο του επιχείρησης. \\

\noindent
Μέσω αυτής έχουν καταγραφεί τα βασικά χαρακτηριστικά του καταλύματος:
	\begin{itemize}
		\item Αριθμός εταιριών με τις οποίες συνεργάζεται ξενοδοχείο.
		\item Τρόποι εύρεσης της επιχείρησης. 
		\item Τρόπους επικοινωνίας πελάτη με την επιχείρηση για να γίνει κράτηση.
		\item Τρόπους επικοινωνίας επιχείρησης με τους συνεργάτες.
		\item Τρόπους καταγραφής προμηθειών.
	\end{itemize}

\hspace{1cm}\\
\noindent
Επίσης, έγινε ενημέρωση για τις παροχές που προσφέρει το κατάλυμα
\begin{itemize}
	\item Πρωινό.
	\item Υπηρεσίες δωματίου (Φαγητό).
 	\item Ποδήλατα για μετακίνηση μέσα στην πόλη.
 	\item Εφαρμογές για ενοικίαση οχημάτων, προτάσεις εστιατορίων και αξιοθέατων.
 	\item Παροχές που αντιστοιχούν στην κατηγορία καταλυμάτων που βρίσκεται.
\end{itemize} 

\noindent
Δόθηκε ιδιαίτερη βάση στις παροχές που προσφέρει το κατάλυμα και ειδικότερα σε αυτές που ανεβάζουν την κατηγορία ενός καταλύματος. Πιο συγκεκριμένα, διαθέτει έξυπνες λάμπες, smart tv, έξυπνους θερμοστάτες και tablet για την χρήση από του πελάτες.\\

\noindent
Όσον αφορά τα προβλήματα που αντιμετωπίζει η επιχείρηση. Αναφέρθηκαν προβλήματά μεταξύ της επικοινωνίας των πελατών και της υποδοχής καθώς πέραν του τηλεφώνου δεν υπάρχει άλλος τρόπος άμεσης επικοινωνίας με αποτέλεσμα πολλοί από τους πελάτες να επικοινωνούν με την εκπρόσωπο που είναι και η υπάλληλος της υποδοχής μέσω εφαρμογών όπως το Whats app και το viber. Αυτό αρχικά χαρακτηρίστηκε ως αντιεπαγγελματικό και επίσης αρκετά κουραστικό για την υπάλληλο της υποδοχής καθώς τα μηνύματα έρχονται και σε ώρες εκτός του ωραρίου. \\

\noindent
Ακόμη, εκφράστηκε η θέληση να γίνεται ουσιαστική χρήση των tablet από τους πελάτες καθώς τώρα δεν χρησιμοποιούνται. Επίσης, αναφέρθηκε πως πολλές από τις ερωτήσεις που γίνονται στον υπάλληλο υποδοχής είναι ερωτήσεις που αναφέρονται ήδη στις παρεχόμενες εφαρμογές με πληροφορίες.

\newpage
\noindent
Γνωρίζοντας πλέον τις αδυναμίες της επιχείρησης, οι συμμετάσχοντες αποφάσισαν να υλοποιήσουν ένα σύστημα το οποίο θα διευκολύνει την επικοινωνία μεταξύ πελάτη και υποδοχής για θέματα του ξενοδοχείου όπως το room service και ταυτόχρονα θα παρέχει πληροφορίες για την πόλη όπως προτάσεις για συγκεκριμένες επιχειρήσεις ενώ θα αφαιρεί αρκετό φόρτο εργασίας από τον υπάλληλο υποδοχής καθώς όπως αναφέρθηκε πολλές από τις ερωτήσεις θα είναι ήδη απαντημένες.\\

\noindent
Πιο συγκεκριμένα, το σύστημα θα επιτρέπει στον χρήστη να βρίσκει πληροφορίες για τα παρακάτω
\begin{itemize}
	\item Προτεινόμενα μέρη για επίσκεψη/δραστηριότητες (σύμφωνα με προτάσεις της γραμματείας).
	\item Προτεινόμενες επιχειρήσεις (σύμφωνα με προτάσεις της γραμματείας).
	\item Συχνές ερωτήσεις σχετικά με την πόλη.
	\item Δυνατότητα χρήσης υπηρεσιών δωματίου όπως για παράδειγμα πρωινό στο δωμάτιο ή αγορά κάποιου άλλου φαγητού ή ποτού όπως κάποιου κρασιού.
	\item Δυνατότητα επικοινωνίας με την υποδοχή σε περίπτωση που οι παρεχόμενες πληροφορίες/δυνατότητες του συστήματος δεν επαρκούν.
\end{itemize} 

\noindent
Οι συμμετάσχοντες θεωρούν πως η υλοποίηση αυτού του συστήματος θα έχει ως αποτέλεσμα την αύξηση της παραγωγικότητας της γραμματείας καθώς δεν θα υπάρχουν διασπάσεις προσοχής λόγω των συνεχών μηνυμάτων και επιπλέον η εμπειρία διαμονής θα γίνει πολύ πιο εύκολη και πολύ πιο ευχάριστη.\\ 

\noindent
Όσον αφορά τα προβλήματα που έχουν παρουσιαστεί, μέχρι τώρα οι συμμετάσχοντες δεν έχουν συναντήσει κάποιο πρόβλημα πέραν της αρχικής επικοινωνίας με τον πελάτη.\\

\noindent
Τέλος, όσον αφορά τον καταμερισμό εργασίας και εφόσον η εργασία βρίσκεται σε αρχικό στάδιο, αρχικά χωρίστηκε τό πρώτο κομμάτι της εργασίας (Ενότητα 2) στους συμμετάσχοντες και στη συνέχεια, θα γίνει διαχωρισμός μίας μεγαλύτερης υποενότητας στον κάθε ένα, ώστε να γίνεται με αποδοτικό τρόπο η αναφορά.