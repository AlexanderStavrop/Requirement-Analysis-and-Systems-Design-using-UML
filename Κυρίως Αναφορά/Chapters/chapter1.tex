\chapter{Εισαγωγή}

Στην ακόλουθη αναφορά παρουσιάζεται ένα πληροφοριακό σύστημα που 
αναλύθηκε και σχεδιάστηκε για την επιχείρηση Balance Hotel στα Χανιά. 
Εμπνεόμενοι από το όνομα του ξενοδοχείου, βασικός στόχος ήταν η επίτευξη 
της ισορροπίας μεταξύ της χρήσης της τεχνολογίας αλλά και της ανθρώπινης 
αλληλεπίδρασης ώστε να βελτιωθεί η διαμονή των πελατών καθώς και η 
καθημερινότητα των εργαζομένων.\\ 

\noindent
Η δεύτερη ενότητα παρέχει αναλυτικές πληροφορίες για την επιχείρηση 
και στην συνέχεια αναφέρεται στο υπάρχον σύστημα δίνοντας μία γενική 
περιγραφή. Έπειτα, αναλύεται σε βάθος η δομή και η αρχιτεκτονική του, τα 
θετικά και τα αρνητικά του, οι προοπτικές και οι κίνδυνοι και, τέλος, οι 
δυνατότητες και οι περιορισμοί του νέου συστήματος.\\

\noindent
Στην τρίτη ενότητα περιγράφονται οι – λειτουργικές και μη – απαιτήσεις 
του καινούριου συστήματος. Ενδεικτικά, μερικές από αυτές είναι οι είσοδοι 
και οι έξοδοι καθώς και τα τεχνικά χαρακτηριστικά.\\

\noindent
Η τέταρτη ενότητα αναφέρει μία λίστα από μετρήσιμα κριτήρια αξιολόγησης 
και έπειτα παρουσιάζει αναλυτικά δύο εμπορικά πακέτα για τα οποία τονίζει 
τα πλεονεκτήματα και τα μειονεκτήματά τους. Έπειτα, γίνεται μία οικονομική
ανάλυση για όλες τις πιθανές επιλογές, συμπεριλαμβανομένου και του δικού 
μας συστήματος, και προτείνεται η βέλτιστη λύση.\\

\noindent
Η πέμπτη, και τελευταία, ενότητα της εργασίας είναι αφιερωμένη αποκλειστικά 
σε διαγράμματα τα οποία βασίζονται σε αρκετά από όσα έχουν αναφερθεί στα 
προηγούμενα κομμάτια της αναφοράς και αναλύουν το σύστημα με 
επιστημονικά ορθό τρόπο.


\section{Εxecutive Summary}

Η ακόλουθη εργασία εκπονήθηκε στο πλαίσιο του μαθήματος «Σχεδίαση και 
Ανάπτυξη Πληροφοριακών Συστημάτων» όπου και κληθήκαμε να διαλέξουμε 
μία εταιρεία για να συνεργαστούμε. Απώτερος σκοπός της συνεργασίας είναι η
 ανάλυση και ο σχεδιασμός ενός πληροφοριακού συστήματος που θα 
 ανταποκρίνεται στις ανάγκες του «πελάτη». Μοναδικός περιορισμός ήταν ο 
 τομέας δραστηριοποίησής του όπου στην προκειμένη περίπτωση θα έπρεπε 
 να προέρχεται από τον χώρο του τουρισμού και ειδικότερα να πρόκειται για 
 επιχείρηση καταλυμάτων. Πρωταρχικός στόχος ήταν η εύρεση του υποψήφιου 
 «πελάτη» με τον οποίο θα έπρεπε να συναντηθούμε ώστε να συζητήσουμε τις 
 ανάγκες του. Την επίτευξη αυτού διαδέχθηκε μία λεπτομερής ανάλυση των 
 συστημάτων που θα υλοποιούνταν όπου έπειτα από αρκετή συζήτηση 
 καταλήξαμε σε δύο διαφορετικά αλλά εξίσου χρήσιμα και αναγκαία για ένα 
 ξενοδοχείο. Το πρώτο εξ αυτών στοχεύει στην βελτίωση της διαμονής των 
 πελατών χρησιμοποιώντας τα ήδη υπάρχοντα tablets ενώ το δεύτερο έχει ως 
 κύριο μέλημά του να διευκολύνει την καθημερινή δουλειά των καμαριέρων 
 κάνοντας παράλληλα και πιο αποδοτικό το καθάρισμα των δωματίων.
Το τελικό «προϊόν», λοιπόν, βασίζεται σε υφιστάμενες παροχές τις οποίες και 
εκμεταλλεύεται δημιουργικά, βελτιώνοντας την εμπειρία διαμονής του πελάτη, 
διευκολύνοντας την εργασία των υπαλλήλων και αποφέροντας κέρδος στον
ιδιοκτήτη. 
