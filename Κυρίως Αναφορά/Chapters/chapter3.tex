\chapter{Ανάλυση Απαιτήσεων Συστήματος}

\section{Λειτουργικές Απαιτήσεις}
\subsubsection{Έξοδοι του Συστήματος}
Το σύστημα χρησιμοποιεί τρία διαφορετικά είδη εξόδου ώστε να παρέχει στους χρήστες τις κατάλληλες
πληροφορίες για τις αντίστοιχες διαδικασίες. \\

\noindent 
Ο πρώτος τύπος εξόδου είναι η άμεση πρόσβαση σε πληροφορίες που ζήτησε ο χρήστης. Κύριος και 
μοναδικός αποδέκτης αποτελεί ο πελάτης καθώς αυτός ο τύπος εξόδου εμφανίζεται στις περιπτώσεις 
όπου ο πελάτης αιτηθεί  είτε πληροφορίες σχετικά με την πόλη των χανιών είτε τον  κατάλογο των 
φαγητών και των ποτών που διαθέτει το κατάλυμα. Οι πληροφορίες αυτές θα επιτρέψουν στον πελάτη 
να έχει  περισσότερες επιλογές διασκέδασης εκτός δωματίου αλλά και  περισσότερες επιλογές 
χαλάρωσης εντός, με αποτέλεσμα η διαμονή του στα Χανιά να γίνει πιο ευχάριστη.\\ 

\noindent
Ο δεύτερος τύπος εξόδου είναι η εμφάνιση νέου μικρού παραθύρου(pop-up window), συνοδευόμενο από 
μήνυμα σχετικό με  την διαδικασία. Αποδέκτης αυτής της εξόδους μπορεί να είναι είτε ο πελάτης είτε 
μία από τις καμαριέρες. Όσον αφορά τον πελάτη, αυτού του είδους η έξοδος  παράγεται στην περίπτωση 
όπου αιτηθεί κάποια από τις παρεχόμενες υπηρεσίες δωματίου. Σε αυτή την περίπτωση, ο πελάτης 
γνωρίζει πως το αίτημα του έχει αποσταλεί και αναμένει την έγκριση ή την απόρριψη του. Όσον αφορά 
τις καμαριέρες, η έξοδος αυτή παράγεται στις περιπτώσεις όπου  συμπληρώνεται ο αριθμός δωματίων 
στα οποία έγινε ο απαραίτητος έλεγχος για απώλειες αντικειμένων καθώς και οι απαραίτητες διαδικασίες 
για την καθαριότητά του. Μέσω αυτής της εξόδου, ο υπάλληλος ενημερώνεται πως έχει ολοκληρώσει 
αυτό το κομμάτι των αρμοδιοτήτων του και μπορεί να συνεχίσει με τις υπόλοιπες. Ακόμα, αυτού του 
είδους η έξοδος παράγεται στις περιπτώσεις όπου κάποιο γεύμα πρέπει να παραδοθεί σε συγκεκριμένο 
δωμάτιο. Σε αυτή την περίπτωση, η καμαριέρα ενημερώνεται για το αίτημα παράδοσης και σε περίπτωση 
έγκρισης αυτού, εμφανίζονται οι περαιτέρω πληροφορίες του αντίστοιχου δωματίου με τις οποίες 
μπορεί να κάνει την  παράδοση εύκολα και στον σωστό χρόνο.\\

\noindent
Τέλος, ο τρίτος τύπος εξόδου είναι η εμφάνιση διαφορετικών γραφικών εικονιδίων για τις διαδικασίες 
"εστάλη", "διαβάστηκε" καθώς και για την απάντηση. Αυτού του τύπου οι έξοδοι χρησιμοποιούνται στις 
προσωπικές συνομιλίες μεταξύ υπάλληλου υποδοχής και πελάτη, στις περιπτώσεις που κάποιος από 
τους δυο αιτηθεί προσωπική συνομιλία με τον άλλο. Πιο συγκεκριμένα, όταν ένα μήνυμα  έχει σταλθεί 
από τον έναν χρήστη στον άλλο, ένα μικρό αχνό τικ θα εμφανίζεται κάτω από αυτό, ενώ όταν διαβαστεί, 
θα εμφανίζονται 2, το ένα δίπλα στο άλλο, με έντονη απόχρωση ώστε να είναι εμφανή.  Τέλος, έξοδος
θεωρείται και η απάντηση που θα δεχτή ο κάθε χρήστης η οποία θα εμφανίζεται σε διαφορετικό πλαίσιο 
κειμένου. Με την χρήση αυτών των συμβόλων, οι χρήστες καταλαβαίνουν εύκολα και γρήγορα σε τι 
στάδιο βρίσκεται η επικοινωνία μεταξύ τους.

\subsubsection{Είσοδοι του Συστήματος}

Οι είσοδοι του συστήματος χωρίζονται ανάλογα με το είδος του χρήστη. Λόγω της φύσης του 
συστήματος, οι χρήστες του συστήματος είναι οι πελάτες του καταλύματος, ο υπεύθυνος υποδοχής 
καθώς και οι καμαριέρες και ο κάθε ένας έχει τη δυνατότητα εκτέλεσης διαφορετικών διαδικασιών και ως
αποτέλεσμα, εισαγωγή διαφορετικών εισόδων σε αυτό. \\

\noindent
Οι πελάτες, αλληλεπιδρούν με το σύστημα μέσω διάφορων διαδικασιών. Ο πιο απλός  τρόπος είναι η 
επιλογή μίας από τις παρεχόμενες λειτουργίες ενώ κάποιες από αυτές απαιτούν περαιτέρω πληροφορίες 
από το χρήστη.  Πιο συγκεκριμένα, στον πελάτη δίνεται η δυνατότητα να διαβάσει πληροφορίες για την 
πόλη  των χανιών, να ζητήσει κάποια από τις παρεχόμενες υπηρεσίες δωματίου καθώς και η δυνατότητα 
επικοινωνίας με τον υπεύθυνο υποδοχής μέσω chat. Όσον αφορά τις πληροφορίες, δεν ζητείται κάποια 
άλλη πληροφορία από τον χρήστη, ενώ όσον αφορά την αίτηση υπηρεσιών δωματίου, από τον χρήστη 
ζητείται η προσθήκη της ημερομηνίας και της ώρας κατά την οποία ο πελάτης θέλει να παραλάβει το 
γεύμα του. Αυτές οι πληροφορίες εισάγονται από τον χρήστη  μέσω πληκτρολογίου το οποίο εμφανίζεται 
στην οθόνη του. Ακόμα, σε περίπτωση που ο πελάτης θελήσει να επικοινωνήσει με την γραμματεία, για 
την εισαγωγή κειμένου εμφανίζεται πληκτρολόγιο ώστε να μπορεί να πληκτρολογήσει το μήνυμα. 
Εφόσον οι πελάτες θα αλληλεπιδρούν με το σύστημα μέσω συσκευών tablet, όλες οι παραπάνω
διαδικασίες γίνονται με της χρήση αφής μέσω της οθόνης του. \\

\noindent
Στον υπάλληλο υποδοχής, δίνεται η δυνατότητα επικοινωνίας με τους πελάτες.
Πιο συγκεκριμένα, ο υπάλληλος μπορεί να επιλέξει με ποιο από τα δωμάτια του ξενοδοχείου επιθυμεί να 
επικοινωνήσει  και στη συνέχεια του δίνεται η δυνατότητα αποστολής μηνύματος. Και πάλι, η 
αλληλεπίδραση γίνεται μέσω αφής. \\

\noindent
Τέλος, στις καμαριέρες, δίνεται η επιλογή  καταγραφής των δωματίων που έχει επισκεφτεί. Πιο
συγκεκριμένα, ζητείται από τον χρήστη η εισαγωγή του αριθμού δωματίου και στη συνέχεια η 
συμπλήρωση  μίας λίστας με τα αντικείμενα που αναμένεται να  υπάρχουν στο δωμάτιο.  Και πάλι, η 
αλληλεπίδραση γίνεται μέσω συσκευής tablet και ως  αποτέλεσμα μέσω αφής.


\subsubsection{Προτεραιότητες και Διαδικασίες του Συστήματος}
Οι προβλεπόμενες διαδικασίες που περιγράφονται παρακάτω, απεικονίζονται στο διάγραμμα \ref{} \\

\noindent
Οι διαδικασίες που μπορεί να εκτελέσει ο πελάτης είναι 3. Πρώτη, είναι η επιλογή ανάγνωσης  συχνών 
ερωτήσεων στις οποίες μπορεί να ενημερωθεί για δραστηριότητες, αξιοθέατα και μαγαζιά τα οποία 
μπορεί να επισκεφτεί καθώς και συνεργαζόμενες επιχειρήσεις με το Balance Hotel που προσφέρουν μία
πληθώρα υπηρεσιών. Για την διαδικασία αυτή το μόνο που απαιτείται από τον χρήστη είναι η επιλογή 
της, στο αρχικό μενού.  Δεύτερη, είναι η αίτηση για παροχή υπηρεσιών δωματίου όπως βραδινό γεύμα ή 
πρωινό. Για την διαδικασία αυτή,  απαιτείται η  επιλογή της στο αρχικό μενού και στην συνέχεια η 
επιλογή του αντίστοιχου γεύματος μέσα από μία λίστα με τα παρεχόμενα γεύματα καθώς και η εισαγωγή 
ημερομηνίας και ώρας παράδοσης.  Στη συνέχεια, αποστέλλεται το αίτημα και ο πελάτης αναμένει έως
ότου το αίτημα του απαντηθεί. Η απάντηση δίνεται από μία από τις καμαριέρες που βρίσκονται σε \\
βάρδια και σε περίπτωση που το αίτημα αφορά ημερομηνία διαφορετική από την σημερινή το αίτημα 
καταγράφεται ενώ σε περίπτωση που το αίτημα αφορά την σημερινή ημερομηνία, εξετάζεται εάν οι 
προμήθειες επαρκούν και ανάλογα με αυτές απορρίπτεται ή εγκρίνεται. Σε κάθε περίπτωση, ο πελάτης 
λαμβάνει το αντίστοιχο μήνυμα που τον ενημερώνει για την διαδικασία. Τρίτη και τελευταία διαδικασία
αποτελεί η αίτηση συνομιλίας μέσω chat με τον υπεύθυνο υποδοχής. 






















\clearpage
\section{Μή Λειτουργικές Απαιτήσεις}
