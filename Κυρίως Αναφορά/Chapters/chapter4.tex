\chapter{Επιλογές και Προτάσεις}
\section{Κριτήρια Αξιολόγησης Επιλογών}
Λαμβάνοντας υπόψιν τις λειτουργικές και μη λειτουργικές απαιτήσεις του συστήματος που αναλύθηκαν στις ενότητες \ref{3.1} και \ref{3.2},
προέκυψαν τα παρακάτω μετρήσιμα κριτήρια ώστε να μπορεί ο πελάτης να συγκρίνει τις διαθέσιμες επιλογές του και να προβεί στην καλύτερη 
δυνατή λύση. Ειδικότερα, τα κριτήρια αυτά είναι:
\begin{enumerate}
	\item Ταχεία και εύκολη εξυπηρέτηση των πελατών
	\item  Φιλικό γραφικό περιβάλλον τόσο για τους πελάτες αλλά και το προσωπικό
	\item  Χρόνος εκμάθησης για το προσωπικό
	\item Απρόσκοπτη πρόσβαση καθ’ όλη την διάρκεια της ημέρας
	\item Βέλτιστη αξιοποίηση προσωπικού της επιχείρησης
	\item Χρόνος ανάπτυξης
	\item Χρόνος εγκατάστασης
	\item Κόστος ανάπτυξης
	\item Κόστος εγκατάστασης
	\item Κόστος συντήρησης
	\item Περιθώριο αύξησης του κέρδους
	\item Ασφάλεια συστήματος τόσο από κυβερνοεπιθέσεις αλλά και από φυσικά φαινόμενα
\end{enumerate}

\section{Εμπορικά Πακέτα Λογισμικού}
\section{Επιλογή Ανάπτυξης Νέου Συστήματος}
\section{Τελική Πρόταση}

