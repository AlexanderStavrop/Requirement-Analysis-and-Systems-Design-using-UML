\chapter{Παραρτήματα}
		
\section{Διάρθρωση Ομάδας και Κατανομή Αρμοδιοτήτων}
\section{Ερωτηματολόγιο Εντοπισμού Απαιτήσεων Λειτουργίας Συστήματος}
\textbf{1) Πείτε μας κάποιες βασικές πληροφορίες όπως έτος ίδρυσης, αριθμός 
	υπαλλήλων που απασχολούνται και συνεργαζόμενες εταιρείες.} \\

\noindent
\textbf{2) Πείτε μας με ποιούς τρόπους μπορεί ένας πελάτης να επικοινωνήσει
	με τον κατάλυμα ώστε να ενημερωθεί για τις παροχές του και να γίνει κράτηση
	σε αυτό.} \\

\noindent
\textbf{3) Πείτε μας ποιός είναι ο μέγιστος αριθμός πελατών που μπορεί να φιλοξενήσει 
	το κατάλυμα και πόσα τα δωμάτια αυτά.} \\

\noindent
\textbf{4) Πότε παρουσιάζεται η μεγαλύτερη κινητικότητα στο ξενοδοχείο.} \\ 

\noindent
\textbf{5) Πως αποθηκεύονται οι απαραίτητες πληροφορίες κάθε πελάτη.} \\

\noindent
\textbf{6) Με ποιούς τρόπους μπορεί ένας πελάτης μπορεί  να επικοινωνήσει 
	με τον υπεύθυνο ή να αιτηθεί κάποια από τις παρεχόμενες παροχές.} \\

\noindent
\textbf{7) Πείτε μας με ποιές παροχές παρέχονται στους πελάτες κατά την διαμονή 
	του στο κατάλυμα.} \\

\noindent
\textbf{8) Πείτε μας με ποιούς τρόπους μπορεί το κατάλυμα να επικοινωνήσει με τις 
	συνεργαζόμενες εταιρείες.} \\

\noindent
\textbf{9) Πείτε μας κάποιες βασικές λειτουργίες που εκτελούνται με το υπάρχον 
	σύστημα.} \\

\noindent
\textbf{10) Κατά την διάρκεια της πανδημίας, παρουσιάστηκαν διαφορές στην
	λειτουργία του καταλύματος.} \\

\noindent
\textbf{11) Τι λειτουργίες θα θέλατε να κάνει το σύστημα και με τι βαθμό 
	προτεραιότητας.} \\


\section{Βιβλιογραφία και Πηγές Πληροφοριών}
\section{Πρακτικά Συναντήσεων}