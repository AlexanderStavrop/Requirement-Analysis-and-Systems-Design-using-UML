\chapter{Παραρτήματα}
		
\section{Διάρθρωση Ομάδας και Κατανομή Αρμοδιοτήτων}
\section{Ερωτηματολόγιο Εντοπισμού Απαιτήσεων Λειτουργίας Συστήματος}
\textbf{1) Πείτε μας κάποιες βασικές πληροφορίες όπως έτος ίδρυσης, αριθμός 
	υπαλλήλων που απασχολούνται και συνεργαζόμενες εταιρείες.} \\ \\
\noindent
Η επιχείρηση ξεκίνησε την λειτουργία της τον χειμώνα του 2017, απασχολεί κατά 
το μέγιστο 5 άτομα,(ποιοι είναι οι ρόλοι) και συνεργάζεται με μία επιχείρηση
για την προμήθεια καθαριστικών, 3 για τον ανεφοδιασμό τροφίμων, και μία για 
τα κλινοσκεπάσματα και τα λευκά είδη. \\ \\

\noindent
\textbf{2) Πείτε μας με ποιούς τρόπους μπορεί ένας πελάτης να επικοινωνήσει
	με τον κατάλυμα ώστε να ενημερωθεί για τις παροχές του και να γίνει κράτηση
	σε αυτό.} \\ \\
\noindent
Αυτή τη στιγμή το κατάλυμα δεν διαθέτει κάποιου είδους site αλλά ο πελάτης
μπορεί να το βρει μέσω πλατφορμών όπως 	E-Booking και Xpedia και να κάνει
κράτηση είτε μέσω site είτε μέσω τηλεφώνου. \\ \\ 
  
\noindent
\textbf{3) Πείτε μας ποιός είναι ο μέγιστος αριθμός πελατών που μπορεί να φιλοξενήσει 
	το κατάλυμα και πόσα τα δωμάτια αυτά.} \\ \\
\noindent
Το κατάλυμα διαθέτει 9 δωμάτια εκ των οποίων τα 8 είναι δίκλινα και το ένα τρίκλινο.
και ως αποτέλεσμα έχει τη δυνατότητα να φιλοξενεί έως και 19 άτομα.\\ \\

\noindent
\textbf{4) Πότε παρουσιάζεται η μεγαλύτερη κινητικότητα στο ξενοδοχείο.} \\ \\
\noindent
Μεγαλύτερη κινητικότητα παρατηρείται κατά τους καλοκαιρινούς μήνες όπου 
και ο αριθμός των τουριστών, είναι ο μεγαλύτερος.\\ \\

\noindent
\textbf{5) Ποιές είναι οι απαραίτητες πληροφορίες για κάθε πελάτη.} \\
\noindent
Κατά κύριο λόγο, οι πληροφορίες που ζητούνται από τον πελάτη είναι 
το όνομα επίθετο και η ημερομηνία άφιξης. \\ \\

\noindent
\textbf{6) Με ποιό τρόπο αποθηκεύονται αυτές οι πληροφορίες.} \\
\noindent
Οι πληροφορίες αυτές βρίσκονται σε ηλεκτρονική μορφή στον κεντρικό υπολογιστή
του καταλύματος ως backup αλλά και σε φυσική μορφή καθώς έτσι είναι το νόμιμο. \\ \\

\noindent
\textbf{7) Με ποιούς τρόπους μπορεί ένας πελάτης μπορεί  να επικοινωνήσει 
	με τον υπεύθυνο ή να αιτηθεί κάποια από τις παρεχόμενες παροχές.} \\
\noindent
Ο πελάτης έχει τη δυνατότητα να επικοινωνήσει με την υποδοχή μέσω τηλεφώνου
είτε με φυσική παρουσία, εάν κατέβει στην υποδοχή.\\ \\

\noindent
\textbf{8) Πείτε μας με ποιές παροχές παρέχονται στους πελάτες κατά την διαμονή 
	του στο κατάλυμα.} \\ \\
\noindent
Στον πελάτη, στην παρούσα φάση, στον πελάτη παρέχονται οι εξής παροχές ???????
?????????????????? \\ \\

\noindent
\textbf{9) Πείτε μας με ποιούς τρόπους μπορεί το κατάλυμα να επικοινωνήσει με τις 
	συνεργαζόμενες εταιρείες.} \\ \\
\noindent
Στην παρούσα φάση, το κατάλυμα επικοινωνεί με τις συνεργαζόμενες επιχειρήσεις
μέσω τηλεφώνου ή Email.\\ \\

\noindent
\textbf{10) Πείτε μας κάποιες βασικές λειτουργίες που εκτελούνται με το υπάρχον 
	σύστημα.}\\ \\
Στο παρόν σύστημα, εκτελούνται οι εξής λειτουργίες. Πρώτον ο πελάτης μπορεί να
παραγγείλει κάποιο γεύμα στο δωμάτιο και δεύτερον γίνεται η καταγραφή των 
απωλειών από κάθε δωμάτιο μετά το τέλος διαμονής ενός πελάτη.\\ \\ 

\noindent
\textbf{11) Κατά την διάρκεια της πανδημίας, παρουσιάστηκαν διαφορές στην
	λειτουργία του καταλύματος.} \\ \\
Κατά την διάρκεια της πανδημίας, έγινε υποχρεωτική και καταγραφή της χώρας
προέλευσης του πελάτη για τις περιπτώσεις όπου υπάρχει κρούσμα covid. \\ 

\noindent
\textbf{12) Τι λειτουργίες θα θέλατε να κάνει το σύστημα και με τι βαθμό 
	προτεραιότητας.} \\ \\
Βασικές λειτουργίες του συστήματος θα είναι η λίστα με τις συχνές ερωτήσεις 
καθώς και η επικοινωνία μέσω chat με τον πελάτη, ενώ αν είναι εφικτό να υπάρχει
η λειτουργία μέσω της οποία θα μπορεί να κάνει παραγγελίες και τελευταία όσον 
αφορά τις προτεραιότητες, είναι η δυνατότητα καταγραφής και αναφοράς απωλειών
από τα δωμάτια. 

\section{Βιβλιογραφία και Πηγές Πληροφοριών}
\section{Πρακτικά Συναντήσεων}