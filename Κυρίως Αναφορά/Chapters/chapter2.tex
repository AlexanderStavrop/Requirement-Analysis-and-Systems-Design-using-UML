\section{Περιγραφή Πλαισίου Έργου}

	\subsection{Πληροφορίες Πελάτη}
	\subsection{Περιγραφή Τρέχοντος Συστήµατος Πελάτη}
	\subsection{Αρχιτεκτονική και Πλατφόρµα Τρέχοντος Συστήµατος}
	\subsection{Πλεονεκτήματα, Αδυναμίες, Ευκαιρίες και Απειλές}
	
	\newpage
	\subsection{Εμβέλεια Έργου και Περιορισμοί Κύκλου Έργου}
		
		\noindent
		Το σύστημα έχει σκοπό να διαχειρίζεται την επικοινωνία μεταξύ υποδοχής και πελάτη κατά την διάρκεια της διαμονής του ενώ παράλληλα να αυξήσει την αποδοτικότητα και την ταχύτητα βασικών εργασιών που συμβαίνουν στα δωμάτια σε τακτά χρονικά διαστήματα.\\
		
		\noindent
		Αρχικά, μέσω του υλοποιούμενου συστήματος ο πελάτης θα έχει την δυνατότητα αλληλεπίδρασης, σε πραγματικό χρόνο, με τον υπεύθυνο του καταλύματος με σκοπό να δεχτεί υπηρεσίες δωματίου όπως το πρωινό στο δωμάτιο σε συγκεκριμένο ώρα ή την παραγγελία κάποιου κρασιού από την κάβα που διαθέτει το κατάλυμα ενώ θα ενημερώνεται σε πραγματικό χρόνο για την επιβεβαίωση ή την απόρριψη του αιτήματος. Ακόμα, θα δίνεται η δυνατότητα εύρεσης αξιοθέατων της πόλης ή επιχειρήσεων όπως εστιατόρια, μαγαζιά με ζωντανή μουσική αλλά και εταιρείες ενοικίασης οχημάτων, σύμφωνα με προτάσεις του ξενοδοχείου. Οι πιο συνηθισμένες ερωτήσεις θα βρίσκονται σε μορφή FAQ (Frequently Asked Questions) μαζί με τις απαντήσεις τους. Τέλος, θα δίνεται η δυνατότητα επικοινωνίας, μέσω της εφαρμογής, με την υποδοχής και πιο συγκεκριμένα με τον υπάλληλο που δουλεύει εκείνη την ώρα, στην περίπτωση που ο πελάτης επιθυμεί να επικοινωνήσει με αυτή. Οι απαντήσεις εμφανίζονται και αυτές στον πελάτη μέσω της εφαρμογής. \\
		
		\noindent
		Όσον αφορά τις λειτουργίες που θα αφορούν για τους υπαλλήλους, θα δίνεται η δυνατότητα στους υπάλληλους του προσωπικό καθαριότητας να αναφέρουν τον αριθμό των δωματίων που ανέλαβαν καθώς και τον αριθμό των αντικειμένων που λείπουν από κάθε δωμάτιο, σε περίπτωση που συμβεί αυτό, ώστε να μπορούν να αναπληρωθούν άμεσα εφόσον έχει τελειώσει η διαμονή του προηγούμενου πελάτη. \\
		
		\noindent
		(Αυτή τη στιγμή, η γενικότερη επικοινωνία με τον υπεύθυνο υποδοχής γίνεται είτε μέσω τηλεφώνου που διαθέτει κάθε δωμάτιο είτε μέσω εφαρμογών κοινωνικής δικτύωσης (ενδεικτικά Viber, Whatsapp). Ειδικότερα, στις περιπτώσεις που δεν χρησιμοποιείται το τηλέφωνο δωματίου, πιθανώς η επικοινωνία να μην είναι πάντα άμεση καθώς ο υπάλληλος με τον οποίο επικοινωνούν να βρίσκεται εκτός ωραρίου. Αυτό αρχικά θεωρείται αντιεπαγγελματικό και ταυτόχρονα αποθαρρυντικό για τον πελάτη καθώς μία απάντηση μπορεί να σταλεί μετά από αρκετές ώρες (όταν είναι και πάλι η βάρδια του συγκεκριμένου υπαλλήλου). Όσον αφορά την απογραφή, γίνεται σε ένα αρχείο excel το οποίο το  διαχειρίζεται ο υπεύθυνος υποδοχής και αποτελεί μία αρκετά χρονοβόρα διαδικασία.) \\
		
		\noindent
		Σε σχέση με τις επιθυμίες του πελάτη, το σύστημα υπερκαλύπτει τις αρχικές απαιτήσεις δηλαδή την βελτίωση της επικοινωνίας μεταξύ υποδοχής και πελατών αλλά και την αξιοποίηση των tablet που μέχρι στιγμής δεν χρησιμοποιούνται ουσιαστικά. Ταυτόχρονα, χρονοβόρες διαδικασίες όπως αυτή της καταμέτρησης των δωματίων που καθαρίστηκαν και των απωλειών που προέκυψαν θα γίνονται με πολύ πιο αποδοτικό τρόπο. \\
		
		\newpage
		\noindent
		Κατά την υλοποίηση ενδεχομένως προκύψουν προβλήματα ως προς την ταχύτητα απόκρισης του συστήματος καθώς είναι απαραίτητο η επικοινωνία να είναι άμεση και απροβλημάτιστη. Για την εξασφάλιση της ομαλής λειτουργίας πιθανώς να απαιτείται αναβάθμιση των πληροφοριακών συστημάτων του καταλύματος (υπολογιστές, βάση δεδομένων) ή/και χορήγηση κινητών τηλεφώνων νέας γενιάς στους υπαλλήλους που δεν διαθέτουν ώστε  να μπορούν να διαχειρίζονται το σύστημα μέσω της αντίστοιχης εφαρμογής. Παρόλο το μικρό μέγεθος της επιχείρησης, το χρονικό κόστος υλοποίησης ενδέχεται να αυξηθεί κατά (χρονος) συμπεριλαμβανομένου των εργασιών εγκατάστασης, ώστε το σύστημα να ανταποκρίνεται στις προδιαγραφές.
		Ακόμα, είναι απαραίτητο να δοθεί ιδιαίτερη έμφαση στην δημιουργία ενός εύχρηστου γραφικού περιβάλλοντος ώστε η χρήση της εφαρμογής να μην απαιτεί ιδιαίτερες τεχνικές γνώσεις και ως αποτέλεσμα κάθε διαδικασία να γίνεται εύκολα και χωρίς να δυσκολεύει τους χρήστες. Μία τέτοια εφαρμογή, απαιτεί από τον ομάδα που την υλοποιεί, βασικές γνώσεις προγραμματισμού πάνω στις εφαρμογές κινητών αλλά και υπολογιστών ενώ ταυτόχρονα αυξάνει αρκετά τον απαιτούμενο χρόνο υλοποίησης. Ενδεικτικά, μία αντίστοιχη εφαρμογή  απαιτεί (χρόνο) για την δημιουργία της.\\ 
		
		\noindent
		Προς αποφυγή των παραπάνω εξόδων, είναι απαραίτητο να δοθεί ιδιαίτερη προσοχή στις απαιτήσεις υλικών πόρων (Hardware) ώστε να μην είναι μεγάλες. Ακόμα, σημαντικό είναι τα άτομα που θα αναλάβουν την δημιουργία των εφαρμογών να έχουν τις απαραίτητες γνώσεις αλλά και εμπειρία πάνω στον τομέα του. Επίσης, για να υπάρχει το καλύτερο δυνατό αποτέλεσμα είναι σημαντικό ο ιδιοκτήτης του καταλύματος αλλά και οι υπάλληλοι να έχουν την δυνατότητα δοκιμής της εφαρμογής ώστε να παρέχουν το κατάλληλο feedback ώστε το τελικά προϊόν να είναι το καλύτερο πιθανό. Τέλος, κάθε πιθανή καθυστέρηση ενώ δεν εμποδίζει την λειτουργία του καταλύματος, δεν συμβάλει και στην βελτίωσή του κάτι το οποίο μπορεί να θεωρηθεί ως κόστος.
		
		
		    