\noindent
\textbf{Εxecutive Summary}\\

\noindent

Η ακόλουθη εργασία εκπονήθηκε στο πλαίσιο του μαθήματος «Σχεδίαση και 
Ανάπτυξη Πληροφοριακών Συστημάτων» όπου και κληθήκαμε να διαλέξουμε 
μία εταιρεία για να συνεργαστούμε. Απώτερος σκοπός της συνεργασίας είναι η
ανάλυση και ο σχεδιασμός ενός πληροφοριακού συστήματος που θα 
ανταποκρίνεται στις ανάγκες του «πελάτη». Μοναδικός περιορισμός ήταν ο 
τομέας δραστηριοποίησής του όπου στην προκειμένη περίπτωση θα έπρεπε 
να προέρχεται από τον χώρο του τουρισμού και ειδικότερα να πρόκειται για 
επιχείρηση καταλυμάτων.\\

\noindent


Πρωταρχικός στόχος ήταν η εύρεση του υποψήφιου 
«πελάτη» με τον οποίο θα έπρεπε να συναντηθούμε ώστε να συζητήσουμε τις 
ανάγκες του. Την επίτευξη αυτού διαδέχθηκε μία λεπτομερής ανάλυση των 
συστημάτων που θα υλοποιούνταν όπου έπειτα από αρκετή συζήτηση 
καταλήξαμε σε δύο διαφορετικά αλλά εξίσου χρήσιμα και αναγκαία για ένα 
ξενοδοχείο. Το πρώτο εξ αυτών στοχεύει στην βελτίωση της διαμονής των 
πελατών χρησιμοποιώντας τα ήδη υπάρχοντα tablets ενώ το δεύτερο έχει ως 
κύριο μέλημά του να διευκολύνει την καθημερινή δουλειά των καμαριέρων 
κάνοντας παράλληλα και πιο αποδοτικό το καθάρισμα των δωματίων.\\

\noindent

Το τελικό «προϊόν», λοιπόν, βασίζεται σε υφιστάμενες παροχές τις οποίες και 
εκμεταλλεύεται δημιουργικά, βελτιώνοντας την εμπειρία διαμονής του πελάτη, 
διευκολύνοντας την εργασία των υπαλλήλων και αποφέροντας κέρδος στον
ιδιοκτήτη. 